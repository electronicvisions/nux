\chapter{Running software on Nux}
\label{ch:usesw}

Executing a program on Nux in hardware or simulation requires three steps outlined in the next sections.
First, you have to install the cross-compilation toolchain for the PowerISA.
The compilation and linking of the programs for the Nux is described next.
For the actual execution implementation specific software is required to interface with Nux.


\section{Building binutils and the cross-compiler}
Any cross-compiler will do that emits code for the target ``powerpc-eabi'', i.e.\ Power instruction set architecture using the \gls{eabi} \citep{IBM1998}.
There exists however a gcc and binutils version with basic support for the vector instructions at \url{https://github.com/electronicvisions/gcc} and \url{https://github.com/electronicvisions/binutils-gdb}.
An installation script is provided in the Nux repository to download and compile the patched gcc-toolchain for the processor from source (\file{support/gcc/install.sh}).
You may want to change the \code{PREFIX} variable at the top to select the target installation directory.
The gcc with vector instruction support is currently at the version 4.9 and therefore depends on outdated libraries.
The dependencies downloaded in the install script (e.g.\ mpfr, gmp, mpc, \ldots) should be preferred over the most recent versions in order to avoid failing builds due to version incompatibilities.
Refer to \cite{installgcc} for further dependencies and required minimum versions.

The goal of the installation script is to build a minimal gcc and binutils with support for C but without libc.
Support for C++ is added to the compiler while still being experimental.


\section{Compiling programs}

The processor starts execution after reset is released at address 0.
Also, exceptions trigger a control transfer to fixed locations in the range from address 4 to 44.
The final result of compilation is therefore to generate a binary image that can be written to memory with valid code at the correct addresses.
This is achieved by a special linker script \code{elf32nux.x} and the C runtime \code{crt.s}.
An example linker script and C runtime can be found in the libnux at \url{https://github.com/electronicvisions/libnux}.


\subsection{C runtime}
The C runtime is a wrapper for the compiled code with various tasks
\begin{itemize}
	\item initialize the stack pointer
	\item declare interrupt routines
	\item start the execution of the compiled program at the correct symbol
	\item resume into a well defined state at the end of the program
\end{itemize}
An example wrapper is discussed below:
\begin{lstlisting}[language=powerasm]
# crt.s -- part a
.extern start 
.extern reset
.extern _isr_undefined
.extern isr_einput
.extern isr_alignemnt
.extern isr_program
.extern isr_doorbell
.extern isr_fit
.extern isr_dec

.extern stack_ptr_init
\end{lstlisting}
This declares symbols that can be overwritten by other parts of the program.
The \code{start} symbol declared on line 2 is the entry point for the C code.
It is defined by declaring a function \texttt{void start()} in your C code.
The \code{isr_*} symbols on lines 5 to 10 represent interrupt service routines.
The \code{_isr_undefined} symbol is a default handler that is used if no service routine is defined.

\begin{lstlisting}[language=powerasm]
# crt.s -- part b
.text
.extern _start:
reset:
	b __init

	# interrupt jump table
	int_mcheck:    b _isr_undefined
	int_cinput:    b _isr_undefined
	int_dstorage:  b _isr_undefined
	int_istorage:  b _isr_undefined
	int_einput:    b isr_einput
	int_alignment: b isr_alignment 
	int_program:   b isr_program
	int_syscall:   b _isr_undefined
	int_doorbell:  b isr_doorbell
	int_cdoorbell: b _isr_undefined
	int_fit:       b isr_fit
	int_dec:       b isr_dec
\end{lstlisting}
This block is the start of the code section of the binary.
It begins on line 5 with a branch to the \code{__init} symbol defined below.
The linker script ensures that this instruction resides at address 0 of the produced binary image.
After that follows the interrupt jump table on lines 8 through 19.
It consists of branches to interrupt service routines, which are located at the appropriate addresses defined by the hardware implementation.


\begin{lstlisting}[language=powerasm]
# crt.s -- part c
__init:
	# set the stack pointer
	lis 1, stack_ptr_init@h
	addi 1, 1, stack_ptr_init@l
	# start actual program
	bl start

end_loop:
	wait
	b end_loop
\end{lstlisting}
This fragment initializes the stack pointer in general-purpose register 1 as defined by the \gls{eabi} \citep{IBM1998}.
The symbol \code{stack_ptr_init} is defined by the linker script to match the size of the implemented memory.
After initialization on line 7, the wrapper calls the user code at the \code{start} symbol.
If user code returns from the start() function, the loop at the end sends the processor to sleep.
In case of wake-up events, the appropriate service routine will be taken through the interrupt jump table.
On return from the service routine, the branch on line 11 ensures a return to the sleep state.


\subsection{Linker script}
The linker script \code{elf32nux.x} is passed to the linker to configure how the resulting binary is generated.
An example is discussed below:

\begin{lstlisting}
MEMORY {
	ram(rwx) : ORIGIN = 0, LENGTH = 16K
}
\end{lstlisting}
This specifies the memory layout of the implementation.
In the given case we have one memory region with a size of \unit[16]{kib} starting at address 0.
In a tight memory configuration ($ \code{OPT_MEM} = \code{Pu_types::MEM_TIGHT}$) there would be two locations here for data and code.


\begin{lstlisting}
mailbox_size = 4096;
mailbox_end = 0x4000;
mailbox_base = mailbox_end - mailbox_size;
stack_ptr_init = mailbox_base - 8;
\end{lstlisting}
The intention here is to create a reserved memory region at the end called mailbox.
This is used for communication with the environment by software running on Nux.
The stack pointer is initialized to start at lower addresses than the mailbox region.
Note, that the \file{crt.s} wrapper uses the \code{stack_ptr_init} symbol to do the actual initialization of register 1.


\begin{lstlisting}
SECTIONS {
	.text : {
        _isr_undefined = .;

        *crt.o(.text)
        *(.text)

        PROVIDE(isr_einput = _isr_undefined);
        PROVIDE(isr_alignment = _isr_undefined);
        PROVIDE(isr_program = _isr_undefined);
        PROVIDE(isr_doorbell = _isr_undefined);
        PROVIDE(isr_fit = _isr_undefined);
        PROVIDE(isr_dec = _isr_undefined);
	} > ram

	.data : {
		*(.data)
		*(.rodata)
	} > ram

	.bss : {
		*(.bss)
		*(.sbss)
	} > ram

  /DISCARD/ : {
    *(.eh_frame)
  }
}
\end{lstlisting}
This part specifies, where generated code sections should be mapped.
We use only the three sections \code{text} for instructions, \code{data} for data, and \code{bss} for zeroed data.
Line 5 ensures, that the wrapper is positioned at memory location 0, so that reset and interrupt handling work correctly.
Line 3 defines the default interrupt handler \code{_isr_undefined} to be equivalent to the reset vector at address 0.
Lines 8 to 13 connect the default handler to all interrupt service routines that were not defined in user code.


\subsection{Minimum user code}

Minimal C user code consists of just the start() function:
\begin{lstlisting}[language=c]
    void start() {
    }
\end{lstlisting}
\subsection{Generating the binary image}

Generating the final binary image involves three steps:
\begin{enumerate}
    \item Compile source files to object files.
    \item Link object files to executable.
    \item Extract binary image of \code{.text} and \code{.data} sections out of the executable.
\end{enumerate}

The following examples assume an architecture with single main memory as shown in Figure~\ref{fig:ppuinst}.
When using two memories, two images have to be extracted at the end, which requires a different \code{elf32nux.x} file as the one shown above.
The examples also assume, that gcc binaries are visible through the \code{PATH} variable and that it was installed with prefix \code{powerpc-eabi}.


\subsubsection{Compiling}

The assembly wrapper has to be assembled:
\begin{lstlisting}
    $ powerpc-eabi-as -mnux crt.s -o crt.o
\end{lstlisting}
And C-source compiled:
\begin{lstlisting}
    $ powerpc-eabi-gcc -c <sourcefile> -o <objectfile> \
        -mcpu=nux        \
        -ffreestanding   \
        -msdata=none     \
        -mstrict-align   \
        -msoft-float     \
        -mno-relocatable
\end{lstlisting}
The exact meaning of options is given in \cite{gccmanual}.
The option on line 3 tells the compiler to use a ``freestanding'' environment.
From the gcc manual:
\begin{quote}
   A freestanding environment is one in which the standard library may not exist, and program startup may not necessarily be at main.
   The most obvious example is an OS kernel. 
\end{quote}
Line 4 disables the use of the \code{.sdata} (small data) section in generated code.
Refer to \cite{IBM1998} to learn how this section would be used.
Line 5 avoids unaligned memory accesses.
The PowerISA allows embedded implementations to raise an exception on unaligned loads and stores, which is what this implementation does.
Line 6 disables the use of floating point instructions.
Floating point math is instead implemented in software.
Line 7 tells the compiler, that code may not be relocated to a different address at runtime.


\subsubsection{Linking}

The executable is generated by the linker:
\begin{lstlisting}
$ powerpc-eabi-ld crt.o <obj1.o> ... -o <binary.elf> \
        -T elf32nux.x  \
        -static      \
        -nostdlib    \
        -lgcc
\end{lstlisting}
Line 2 tells the linker to use a specified linker script.
Line 3 uses static linking of libraries.
Line 4 disables linking of standard libraries, especially libc (which we do not have).
Line 5 links libgcc.
This is a gcc internal library with routines that the compiler might use instead of directly emitting code for them.
For example, optimized implementations of \code{memcpy} or soft floating point implementations.
See \cite{libgcc} for more information.


\subsubsection{Extracting the binary image}

The objcopy command from binutils allows to extract the raw bits from the executable:
\begin{lstlisting}
$ powerpc-eabi-objcopy -O binary <binary.elf> <image.raw>
\end{lstlisting}


\subsection{Useful commands}

To inspect the resulting image you can use the hexdump program:
\begin{lstlisting}
$ hexdump -C <image.raw>
\end{lstlisting}
It outputs a hex-dump to stdout.

The executable can be inspected using readelf to get information about symbol (variables and functions) locations.
Objdump contains a dissasembler:
\begin{lstlisting}
$ powerpc-eabi-objdump -d <binary.elf>
\end{lstlisting}
The extracted binaries can be disassembled, too, when supplying the relevant information on the architecture and file format:
\begin{lstlisting}
$ powerpc-eabi-objdump \
        -b binary -m powerpc -Mnux --endian=big \
        -d <image.raw>
\end{lstlisting}


\section{Loading and executing programs}

The specifics of how to do this are defined by the system, in which the processor is used.
The binary image generated using objcopy has to be transferred to the memory in the system.
During this time, the processor should be in reset, or you have to know what you are doing.
After loading, reset is released to start the program.
The \code{crt.s} wrapper ensures, that the program goes into the sleep state when the user code terminates, i.e.\ returns from the start() function.



