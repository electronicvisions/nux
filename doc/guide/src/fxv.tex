\section{FXV vector instruction set}

This section lists the implemented instructions of the fixed-point vector functional unit of Nux.
The following rules are used in the notation:
\begin{itemize}
    \item Round brackets indicate the contents of the register selected by the index in brackets.
        So, $(\text{VRA})$ stands for the contents of the register with index VRA.
    \item Subscripts denote the element of the vector using the current type:
        $(\text{VRA})_{i/8}$ is the $i-\text{th}$ halfword element, $(\text{VRA})_{i/16}$ is the $i-\text{th}$ byte element, and $(\text{VRA})_{i/32}$ is the $i-\text{th}$ half-byte element.
    \item Lowercase letters $u, v, w, m$ indicate single-precision elements.
        Uppercase letters $U, V, W, M$ are double-precision elements.
    \item Underlined letters $\underline{u}, \underline{v}$ indicate full single-precision vectors.
    \item VRA, VRB, and VRT refer to vector register indices.
    \item ACC is the contents of the double-precision accumulation register.
    \item VCR is contents of the \unit[32]{bit} vector condition register.
    \item RA, RB refer to general-purpose registers.
\end{itemize}

\subsection{Registers}

\subsubsection{Vector register file (VRF)}

\begin{bytefield}[bitwidth=0.2em]{128}
    \bitheader{0,31,63,95,127} \\
    \bitbox{16}{$\text{VR0}_{0/8}$} & \bitbox{16}{$\text{VR0}_{1/8}$}  & \bitbox{16}{$\text{VR0}_{2/8}$} & \bitbox{16}{$\text{VR0}_{3/8}$} & \bitbox{16}{$\text{VR0}_{4/8}$} & \bitbox{16}{$\text{VR0}_{5/8}$} & \bitbox{16}{$\text{VR0}_{6/8}$} & \bitbox{16}{$\text{VR0}_{7/8}$}\\
    \bitbox{128}{$\vdots$} \\
    \bitbox{128}{VR31} \\
\end{bytefield} \\
The vector register file contains 32 \unit[128]{bit} vector registers, which consist of 8 halfword elements.
Each element can also be used as two byte elements depending on the used instruction.


\subsubsection{Vector condition register (VCR)}

\begin{bytefield}[bitwidth=0.2em]{96}
    \bitheader{0,11,23,35,47,59,71,83,95} \\
    \bitbox{12}{$C_0$}   %
    & \bitbox{12}{$C_1$} %
    & \bitbox{12}{$C_2$} %
    & \bitbox{12}{$C_3$} %
    & \bitbox{12}{$C_4$} %
    & \bitbox{12}{$C_5$} %
    & \bitbox{12}{$C_6$} %
    & \bitbox{12}{$C_7$} \\
\end{bytefield} \\
The vector condition register has one field for each of the 8 halfword elements:

\begin{bytefield}[bitwidth=1em]{12}
    \bitheader{0,3,4,7,8,11} \\
    \bitbox{4}{EQ} & \bitbox{4}{GT} & \bitbox{4}{LT} \\
    {\footnotesize %
    \bitbox{1}{0}   %
    & \bitbox{1}{1} %
    & \bitbox{1}{2} %
    & \bitbox{1}{3} %
    & \bitbox{1}{0} %
    & \bitbox{1}{1} %
    & \bitbox{1}{2} %
    & \bitbox{1}{3} %
    & \bitbox{1}{0} %
    & \bitbox{1}{1} %
    & \bitbox{1}{2} %
    & \bitbox{1}{3} } \\
\end{bytefield} \\
Each field consists of three condition areas for equality (EQ), greater than (GT), and less than (LT).
For each nibble (\unit[4]{bit}) in the vector one condition bit is stored.
Note: nibble-wise condition bits are a leftover from early days.
Currently, there is no way to set or use them individually.
Operations use two or four bits simultaneously for byte or halfword operations.


\subsection{Primitive functions}
Some instructions take only effect for individual elements if a given condition is met.
The condition is determined by the contents of the vector condition register and the condition field in the instruction in the following way:
{\footnotesize
\begin{algorithmic}[1]
    \Function {conditional}{cond, vcr\_field}
        \If { cond = 0 }
            \State \Return true
        \ElsIf { cond = 1 }
            \State \Return vcr\_field.gt
        \ElsIf { cond = 2 }
            \State \Return vcr\_field.lt
        \ElsIf { cond = 3 }
            \State \Return vcr\_field.eq
        \EndIf
    \EndFunction
\end{algorithmic} }


\paragraph{Shift and logical functions}
The $\text{\sc{shift\_left}}(x,n)$ function gives the result of shifting $x$ left by $n$ positions while filling in zeros on the right.
Shifted-out bits are discarded.

The $\text{\sc{shift\_right}}(x,n)$ function gives the result of shifting $x$ right by $n$ positions while filling in zeros on the left.

The $\text{\sc{sign\_extend}}(x)$ function gives the result of sign extending $x$ to the width of the assignee of this function.

The $\text{\sc{bitwise\_and}}(a, b)$ function represents the result of performing a bitwise and between $a$ and $b$.

The $\text{\sc{bitwise\_or}}(a, b)$ function represents the result of performing a bitwise or between $a$ and $b$.

\paragraph{Saturating and fractional arithmetic}
Some instructions use fractional representation usually combined with saturating arithmetics.
The function below defines saturating multiplication of two fractional operands of \unit[$n$]{bit} size.
The result is \unit[$2n$]{bit} wide and shifted to the left to get rid of one superfluous sign bit.
Saturation only occurs if both operands represent $-1$ ($= 100\ldots 0$ in binary).
{\footnotesize
\begin{algorithmic}[1]
    \Function {mult\_sat\_fract}{a, b, n}
        \If { $a = -2^{n-1} \land b = -2^{n-1}$ }
            \State \Return $2^{2*n-1} - 1$
        \Else
            \State $u \gets a \cdot b$
            \State \Return \Call{shift\_left}{$u$, 1}
        \EndIf
    \EndFunction
\end{algorithmic} }

The following function defines saturating addition of two \unit[$n$]{bit} operands.
In case of overflows, the result saturates to the maximum and minimum representable value.
{\footnotesize
\begin{algorithmic}[1]
    \Function {add\_sat}{a, b, n}
        \State $y \gets a + b$
        \If { $y > 2^{n-1} - 1$ }
            \State \Return $2^{n-1} - 1$
        \ElsIf { $y < -2^{n-1}$ }
            \State \Return $-2^{n-1}$
        \Else
            \State \Return $y$
        \EndIf
    \EndFunction
\end{algorithmic} }


\paragraph{Memories and IO}
Function $\text{\sc{mem}}(a)$ represents the \unit[32]{bit }contents of main memory at address $a$.

Function $\text{\sc{io}}(a)$ represents the contents of \gls{io} location $a$.
The width of this result is determined by the width of vectors (\code{OPT_VECTOR_NUM_HALFWORDS}) and the number of slices (\code{OPT_VECTOR_SLICES}).


\begin{multicols}{2}[%
    \subsection{Modulo halfword instructions}%
]

\fxvinst{Multiply accumulate halfword modulo}{fxvmahm}{12}{%
    \For {$0 \le i < 8$}
        \State $u \gets (\text{VRA})_{i/8}$
        \State $v \gets (\text{VRB})_{i/8}$
        \State $W \gets u \cdot v$
        \State $M \gets \text{ACC}_{i/8}$
        
        \State enable $\gets$ \Call{conditional}{C, VCR$_{i/8}$}

        \If { enable }
            \State $(\text{VRT})_{i/8} \gets W + M_i \mod 2^{16}$
        \EndIf

    \EndFor
}{%
    The contents of vector registers VRA and VRB are multiplied as \unit[16]{bit} halfword elements and added to the contents of the \unit[32]{bit} double-precision accumulation register.
    The result is written to vector register VRT modulo $2^{16}$ if the specified condition is met.
}

\fxvinst{Multiply accumulate to accumulator halfword modulo}{fxvmatachm}{44}{%
    \For {$0 \le i < 8$}
        \State $u \gets (\text{VRA})_{i/8}$
        \State $v \gets (\text{VRB})_{i/8}$
        \State $W \gets u \cdot v$
        \State $M \gets \text{ACC}_{i/8}$

        \State enable $\gets$ \Call{conditional}{C, VCR$_{i/8}$}

        \If { enable }
            \State $\text{ACC}_{i/8} \gets W + M_i \mod 2^{32}$
        \EndIf

    \EndFor
}{%
    The contents of vector registers VRA and VRB are multiplied as \unit[16]{bit} halfword elements and added to the contents of the \unit[32]{bit} double-precision accumulation register.
    The result is written to the accumulation register modulo $2^{32}$ if the specified condition is met.
}

\fxvinst{Multiply halfword modulo}{fxvmulhm}{76}{%
    \For {$0 \le i < 8$}
        \State $u \gets (\text{VRA})_{i/8}$
        \State $v \gets (\text{VRB})_{i/8}$

        \State enable $\gets$ \Call{conditional}{C, VCR$_{i/8}$}

        \If { enable }
            \State $(\text{VRT})_{i/8} \gets u \cdot v \mod 2^{16}$
        \EndIf
    \EndFor
}{%
    The contents of vector registers VRA and VRB are multiplied as \unit[16]{bit} halfword elements.
    The result is written to the vector register indicated by VRT modulo $2^{16}$ if the specified condition is met.
}

\fxvinst{Multiply to accumulator halfword modulo}{fxvmultachm}{108}{%
    \For {$0 \le i < 8$}
        \State $u \gets (\text{VRA})_{i/8}$
        \State $v \gets (\text{VRB})_{i/8}$

        \State enable $\gets$ \Call{conditional}{C, VCR$_{i/8}$}

        \If { enable }
            \State $\text{ACC}_{i/8} \gets u \cdot v \mod 2^{32}$
        \EndIf
    \EndFor
}{%
    The contents of vector registers VRA and VRB are multiplied as \unit[16]{bit} halfword elements.
    The result is written to the accumulator modulo $2^{32}$ if the specified condition is met.
}

\fxvinst{Subtract halfword modulo}{fxvsubhm}{332}{%
    \For {$0 \le i < 8$}
        \State $u \gets (\text{VRA})_{i/8}$
        \State $v \gets (\text{VRB})_{i/8}$

        \State enable $\gets$ \Call{conditional}{C, VCR$_{i/8}$}

        \If { enable }
            \State $(\text{VRT})_{i/8} \gets u - v \mod 2^{16}$
        \EndIf
    \EndFor
}{%
    The contents of the vector register indicated by VRB is subtracted from the contents of the register indicated by VRA.
    The result is written to VRT modulo $2^{16}$ if the specified condition is met.
}

\fxvinst{Add accumulator to accumulator halfword modulo}{fxvaddactachm}{364}{%
    \For {$0 \le i < 8$}
        \State $U \gets \Call{sign\_extend}{(\text{VRA})_{i/8}}$
        \State $M \gets ACC_{i/8}$

        \State enable $\gets$ \Call{conditional}{C, VCR$_{i/8}$}

        \If { enable }
            \State $\text{ACC}_{i/8} \gets U + M \mod 2^{32}$
        \EndIf
    \EndFor
}{%
    The contents of vector register VRA is sign extended to double precision and added to the accumulation register.
    The result is written to the accumulator modulor $2^{32}$ if the specified condition is met.
}

\fxvinst{Add and save to accumulator halfword modulo}{fxvaddtachm}{428}{%
    \For {$0 \le i < 8$}
        \State $U \gets \Call{sign\_extend}{(\text{VRA})_{i/8}}$
        \State $V \gets \Call{sign\_extend}{(\text{VRB})_{i/8}}$

        \State enable $\gets$ \Call{conditional}{C, VCR$_{i/8}$}

        \If { enable }
            \State $\text{ACC}_{i/8} \gets U + V$
        \EndIf
    \EndFor
}{%
    The contents of vector registers VRA and VRB are sign extended to double precision.
    If the specified condition is met, their sum is written to the accumulator.
}

\fxvinst{Add accumulator halfword modulo}{fxvaddachm}{396}{%
    \For {$0 \le i < 8$}
        \State $U \gets \Call{sign\_extend}{(\text{VRA})_{i/8}}$
        \State $M \gets \text{ACC}_{i/8}$

        \State enable $\gets$ \Call{conditional}{C, VCR$_{i/8}$}

        \If { enable }
            \State $(\text{VRT})_{i/8} \gets U + M \mod 2^{16}$
        \EndIf
    \EndFor
}{%
    The contents of vector register VRA is sign extended to double precision and added to the accumulator.
    The result is written to vector register VRT modulo $2^{16}$.
}

\fxvinst{Add halfword modulo}{fxvaddhm}{460}{%
    \For {$0 \le i < 8$}
        \State $U \gets \Call{sign\_extend}{(\text{VRA})_{i/8}}$
        \State $V \gets \Call{sign\_extend}{(\text{VRB})_{i/8}}$

        \State enable $\gets$ \Call{conditional}{C, VCR$_{i/8}$}

        \If { enable }
            \State $(\text{VRT})_{i/8} \gets U + V \mod 2^{16}$
        \EndIf
    \EndFor
}{%
    The contents of vector registers indicated by VRA and VRB are added.
    If the specified condition is met, the result is written to the vector register indicated by VRT.
}

\fxvinst{Compare halfword modulo}{fxvcmphm}{300}{%
    \For {$0 \le i < 8$}
        \State $u \gets (\text{VRA})_{i/8}$
        \State $\text{VCR}_{i/8}.eq \gets u = 0$
        \State $\text{VCR}_{i/8}.gt \gets u > 0$
        \State $\text{VCR}_{i/8}.lt \gets u < 0$
    \EndFor
}{%
    The contents of the vector register indicated by VRA is compared to zero and the result written to the vector condition register.
}


\fxvinst{Move to accumulator halfword fractional}{fxvmtach}{15}{%
    \For {$0 \le i < 8$}
        \State enable $\gets$ \Call{conditional}{C, VCR$_{i/8}$}

        \If { enable }
            \State $(\text{ACC})_{i/8} \gets (\text{VRA})_{i/8}$
        \EndIf
    \EndFor
}{%
    If enabled, the contents of elements in register VRA are copied to the accumulator aligned to the right.
}


\fxvinstGPRa{Splat halfword}{fxvsplath}{268}{%
    \For {$0 \le i < 8$}
        \State $u \gets \text{RA} \mod 2^{16}$
        \State $\text{VRT}_{i/8} \gets u$
    \EndFor
}{%
    The lower halfword of general-purpose register RA is copied to each element of VRT.
}
\end{multicols}

\begin{multicols}{2}[%
    \subsection{Modulo byte instructions}%
]

\fxvinst{Multiply accumulate byte modulo}{fxvmabm}{13}{%
    \For {$0 \le i < 16$}
        \State $u \gets (\text{VRA})_{i/16}$
        \State $v \gets (\text{VRB})_{i/16}$
        \State $W \gets u \cdot v$
        \State $M \gets \text{ACC}_{i/16}$
        
        \State enable $\gets$ \Call{conditional}{C, VCR$_{i/16}$}

        \If { enable }
            \State $(\text{VRT})_{i/16} \gets W + M_i \mod 2^{8}$
        \EndIf

    \EndFor
}{%
    The contents of vector registers VRA and VRB are multiplied as \unit[8]{bit} byte elements and added to the contents of the \unit[16]{bit} double-precision accumulation register.
    The result is written to vector register VRT modulo $2^{8}$ if the specified condition is met.
}

\fxvinst{Multiply accumulate to accumulator byte modulo}{fxvmatacbm}{45}{%
    \For {$0 \le i < 16$}
        \State $u \gets (\text{VRA})_{i/16}$
        \State $v \gets (\text{VRB})_{i/16}$
        \State $W \gets u \cdot v$
        \State $M \gets \text{ACC}_{i/16}$

        \State enable $\gets$ \Call{conditional}{C, VCR$_{i/16}$}

        \If { enable }
            \State $\text{ACC}_{i/16} \gets W + M_i \mod 2^{16}$
        \EndIf

    \EndFor
}{%
    The contents of vector registers VRA and VRB are multiplied as \unit[8]{bit} byte elements and added to the contents of the \unit[16]{bit} double-precision accumulation register.
    The result is written to the accumulation register modulo $2^{16}$ if the specified condition is met.
}

\fxvinst{Multiply byte modulo}{fxvmulbm}{77}{%
    \For {$0 \le i < 16$}
        \State $u \gets (\text{VRA})_{i/16}$
        \State $v \gets (\text{VRB})_{i/16}$

        \State enable $\gets$ \Call{conditional}{C, VCR$_{i/16}$}

        \If { enable }
            \State $(\text{VRT})_{i/16} \gets u \cdot v \mod 2^{8}$
        \EndIf
    \EndFor
}{%
    The contents of vector registers VRA and VRB are multiplied as \unit[8]{bit} byte elements.
    The result is written to the vector register indicated by VRT modulo $2^{8}$ if the specified condition is met.
}

\fxvinst{Multiply to accumulator byte modulo}{fxvmultacbm}{109}{%
    \For {$0 \le i < 16$}
        \State $u \gets (\text{VRA})_{i/16}$
        \State $v \gets (\text{VRB})_{i/16}$

        \State enable $\gets$ \Call{conditional}{C, VCR$_{i/16}$}

        \If { enable }
            \State $\text{ACC}_{i/16} \gets u \cdot v \mod 2^{16}$
        \EndIf
    \EndFor
}{%
    The contents of vector registers VRA and VRB are multiplied as \unit[8]{bit} byte elements.
    The result is written to the accumulator modulo $2^{16}$ if the specified condition is met.
}

\fxvinst{Subtract byte modulo}{fxvsubbm}{333}{%
    \For {$0 \le i < 16$}
        \State $u \gets (\text{VRA})_{i/16}$
        \State $v \gets (\text{VRB})_{i/16}$

        \State enable $\gets$ \Call{conditional}{C, VCR$_{i/16}$}

        \If { enable }
            \State $(\text{VRT})_{i/16} \gets u - v \mod 2^{8}$
        \EndIf
    \EndFor
}{%
    The contents of the vector register indicated by VRB is subtracted from the contents of the register indicated by VRA.
    The result is written to VRT modulo $2^{8}$ if the specified condition is met.
}

\fxvinst{Add accumulator to accumulator byte}{fxvaddactacb}{365}{%
    \For {$0 \le i < 16$}
        \State $U \gets \Call{sign\_extend}{(\text{VRA})_{i/16}}$
        \State $M \gets ACC_{i/16}$

        \State enable $\gets$ \Call{conditional}{C, VCR$_{i/16}$}

        \If { enable }
            \State $\text{ACC}_{i/16} \gets U + M \mod 2^{16}$
        \EndIf
    \EndFor
}{%
    The contents of vector register VRA is sign extended to double precision and added to the accumulation register.
    The result is written to the accumulator modulo $2^{16}$ if the specified condition is met.
}

\fxvinst{Add and save to accumulator byte modulo}{fxvaddtacb}{429}{%
    \For {$0 \le i < 16$}
        \State $U \gets \Call{sign\_extend}{(\text{VRA})_{i/16}}$
        \State $V \gets \Call{sign\_extend}{(\text{VRB})_{i/16}}$

        \State enable $\gets$ \Call{conditional}{C, VCR$_{i/16}$}

        \If { enable }
            \State $\text{ACC}_{i/16} \gets U + V$
        \EndIf
    \EndFor
}{%
    The contents of vector registers VRA and VRB are sign extended to double precision.
    If the specified condition is met, their sum is written to the accumulator.
}

\fxvinst{Add accumulator byte modulo}{fxvaddacbm}{397}{%
    \For {$0 \le i < 16$}
        \State $U \gets \Call{sign\_extend}{(\text{VRA})_{i/16}}$
        \State $M \gets \text{ACC}_{i/16}$

        \State enable $\gets$ \Call{conditional}{C, VCR$_{i/16}$}

        \If { enable }
            \State $(\text{VRT})_{i/16} \gets U + M \mod 2^{8}$
        \EndIf
    \EndFor
}{%
    The contents of vector register VRA is sign extended to double precision and added to the accumulator.
    The result is written to vector register VRT modulo $2^{8}$.
}

\fxvinst{Add byte modulo}{fxvaddbm}{461}{%
    \For {$0 \le i < 16$}
        \State $U \gets \Call{sign\_extend}{(\text{VRA})_{i/16}}$
        \State $V \gets \Call{sign\_extend}{(\text{VRB})_{i/16}}$

        \State enable $\gets$ \Call{conditional}{C, VCR$_{i/16}$}

        \If { enable }
            \State $(\text{VRT})_{i/16} \gets U + V \mod 2^{8}$
        \EndIf
    \EndFor
}{%
    The contents of vector registers indicated by VRA and VRB are added.
    If the specified condition is met, the result is written to the vector register indicated by VRT.
}

\fxvinst{Compare byte}{fxvcmpb}{301}{%
    \For {$0 \le i < 16$}
        \State $u \gets (\text{VRA})_{i/16}$
        \State $\text{VCR}_{i/16}.eq \gets u = 0$
        \State $\text{VCR}_{i/16}.gt \gets u > 0$
        \State $\text{VCR}_{i/16}.lt \gets u < 0$
    \EndFor
}{%
    The contents of the vector register indicated by VRA is compared to zero and the result written to the vector condition register.
}


\fxvinst{Move to accumulator byte}{fxvmtacb}{14}{%
    \For {$0 \le i < 16$}
        \State enable $\gets$ \Call{conditional}{C, VCR$_{i/16}$}

        \If { enable }
            \State $(\text{ACC})_{i/16} \gets (\text{VRA})_{i/16}$
        \EndIf
    \EndFor
}{%
    If enabled, the contents of elements in register VRA are copied to the accumulator aligned to the right.
}


\fxvinstGPRa{Splat byte}{fxvsplatb}{269}{%
    \For {$0 \le i < 16$}
        \State $u \gets \text{RA} \mod 2^{8}$
        \State $\text{VRT}_{i/16} \gets u$
    \EndFor
}{%
    The lower byte of general-purpose register RA is copied to each element of VRT.
}
\end{multicols}

\begin{multicols}{2}[%
    \subsection{Saturating, fractional, halfword instructions}%
]

\fxvinst{Multiply accumulate halfword fractional saturating}{fxvmahfs}{28}{%
    \For {$0 \le i < 8$}
        \State $u \gets (\text{VRA})_{i/8}$
        \State $v \gets (\text{VRB})_{i/8}$
        \State $W \gets \Call{mult\_sat\_fract}{u, v, \unit[16]{bit}}$
        \State $M \gets \Call{add\_sat}{\text{ACC}_{i/8}, W, \unit[32]{bit}}$
        
        \State enable $\gets$ \Call{conditional}{C, VCR$_{i/8}$}

        \If { enable }
            \State $(\text{VRT})_{i/8} \gets \Call{shift\_right}{M, 16}$
        \EndIf

    \EndFor
}{%
    Perform multiplication of registers VRA and VRB using fractional representation and saturating on overflows.
    The double width result is added with the contents of the accumulation register again with saturation on overflow.
    The upper 16 bits of the result are returned to register VRT if the condition is met.
    The accumulation register is not modified.
}


\fxvinst{Move to accumulator halfword fractional}{fxvmtachf}{31}{%
    \For {$0 \le i < 8$}
        \State enable $\gets$ \Call{conditional}{C, VCR$_{i/8}$}

        \If { enable }
            \State $(\text{ACC})_{i/8} \gets (\text{VRA})_{i/8} \cdot 2^{16}$
        \EndIf
    \EndFor
}{%
    If enabled, the contents of elements in register VRA are copied to the accumulator aligned to the left.
}


\fxvinst{Multiply accumulate and save to accumulator halfword fractional}{fxvmatachfs}{60}{%
    \For {$0 \le i < 8$}
        \State $u \gets (\text{VRA})_{i/8}$
        \State $v \gets (\text{VRB})_{i/8}$
        \State $W \gets \Call{mult\_sat\_fract}{u, v, \unit[16]{bit}}$
        \State $M \gets \Call{add\_sat}{\text{ACC}_{i/8}, W, \unit[32]{bit}}$
        
        \State enable $\gets$ \Call{conditional}{C, VCR$_{i/8}$}

        \If { enable }
            \State $(\text{ACC})_{i/8} \gets M$
        \EndIf
    \EndFor
}{%
    Perform a multiply-add operation of registers VRA, VRB, and the accumulator using saturation and fractional arithmetics.
    If enabled, the result is stored in the accumulator.
}


\fxvinst{Multiply halfword fractional saturating}{fxvmulhfs}{92}{%
    \For {$0 \le i < 8$}
        \State $u \gets (\text{VRA})_{i/8}$
        \State $v \gets (\text{VRB})_{i/8}$
        \State $W \gets \Call{mult\_sat\_fract}{u, v, \unit[16]{bit}}$
        
        \State enable $\gets$ \Call{conditional}{C, VCR$_{i/8}$}

        \If { enable }
            \State $(\text{VRT})_{i/8} \gets \Call{shift\_right}{M, 16}$
        \EndIf
    \EndFor
}{%
    Perform saturating multiplication in fractional representation of the contents of registers VRA and VRB.
    Store the truncated result to register VRT if enabled.
}


\fxvinst{Multiply and save to accumulator halfword fractional saturating}{fxvmultachfs}{124}{%
    \For {$0 \le i < 8$}
        \State $u \gets (\text{VRA})_{i/8}$
        \State $v \gets (\text{VRB})_{i/8}$
        \State $W \gets \Call{mult\_sat\_fract}{u, v, \unit[16]{bit}}$
        
        \State enable $\gets$ \Call{conditional}{C, VCR$_{i/8}$}

        \If { enable }
            \State $(\text{ACC})_{i/8} \gets M$
        \EndIf
    \EndFor
}{%
    Perform saturating multiplication in fractional representation of the contents of registers VRA and VRB.
    Store the result to the accumulator.
}


\fxvinst{Subtract halfword fractional saturating}{fxvsubhfs}{348}{%
    \For {$0 \le i < 8$}
        \State $U \gets \Call{sign\_extend}{(\text{VRA})_{i/8}}$
        \State $V \gets \Call{sign\_extend}{- (\text{VRB})_{i/8}}$
        \State $W \gets \Call{add\_sat\_fract}{U, V, \unit[32]{bit}}$
        
        \State enable $\gets$ \Call{conditional}{C, VCR$_{i/8}$}

        \If { enable }
            \State $(\text{VRT})_{i/8} \gets \Call{shift\_right}{W, 16}$
        \EndIf
    \EndFor
}{%
    Add the contents of register VRA to the negated contents of register VRB using saturation.
    Store the truncated result to register VRT if enabled.
}


\fxvinst{Add accumulator and save to accumulator halfword fractional}{fxvaddactachf}{380}{%
    \For {$0 \le i < 8$}
        \State $U \gets \Call{sign\_extend}{(\text{VRA})_{i/8}}$
        \State $V \gets (\text{ACC})_{i/8}$
        \State $W \gets \Call{add\_sat\_fract}{U, V, \unit[32]{bit}}$
        
        \State enable $\gets$ \Call{conditional}{C, VCR$_{i/8}$}

        \If { enable }
            \State $(\text{ACC})_{i/8} \gets W$
        \EndIf
    \EndFor
}{%
    Add the sign extended contents of register VRA to the accumulator using saturating arithmetics and store the result to the accumulator.
}


\fxvinst{Add accumulator halfword fractional saturating}{fxvaddachfs}{412}{%
    \For {$0 \le i < 8$}
        \State $U \gets \Call{sign\_extend}{(\text{VRA})_{i/8}}$
        \State $V \gets (\text{ACC})_{i/8}$
        \State $W \gets \Call{add\_sat\_fract}{U, V, \unit[32]{bit}}$
        
        \State enable $\gets$ \Call{conditional}{C, VCR$_{i/8}$}

        \If { enable }
            \State $(\text{VRT})_{i/8} \gets \Call{shift\_right}{W, 16}$
        \EndIf
    \EndFor
}{%
    Add the sign extended contents of register VRA to the accumulator using saturating arithmetics.
    Store the truncated result to register VRT.
}


\fxvinst{Add halfword fractional saturating}{fxvaddhfs}{476}{%
    \For {$0 \le i < 8$}
        \State $U \gets \Call{sign\_extend}{(\text{VRA})_{i/8}}$
        \State $V \gets \Call{sign\_extend}{(\text{VRB})_{i/8}}$
        \State $W \gets \Call{add\_sat\_fract}{U, V, \unit[32]{bit}}$
        
        \State enable $\gets$ \Call{conditional}{C, VCR$_{i/8}$}

        \If { enable }
            \State $(\text{VRT})_{i/8} \gets \Call{shift\_right}{W, 16}$
        \EndIf
    \EndFor
}{%
    Add the contents of registers VRA and VRB using saturating arithmetics.
    Store the result to register VRT.
}


\end{multicols}


\begin{multicols}{2}[%
    \subsection{Saturating, fractional, byte instructions}%
]

\fxvinst{Multiply accumulate byte fractional saturating}{fxvmabfs}{29}{%
    \For {$0 \le i < 16$}
        \State $u \gets (\text{VRA})_{i/16}$
        \State $v \gets (\text{VRB})_{i/16}$
        \State $W \gets \Call{mult\_sat\_fract}{u, v, \unit[8]{bit}}$
        \State $M \gets \Call{add\_sat}{\text{ACC}_{i/16}, W, \unit[16]{bit}}$
        
        \State enable $\gets$ \Call{conditional}{C, VCR$_{i/16}$}

        \If { enable }
            \State $(\text{VRT})_{i/16} \gets \Call{shift\_right}{M, 8}$
        \EndIf

    \EndFor
}{%
    Perform multiplication of registers VRA and VRB using fractional representation and saturating on overflows.
    The double width result is added with the contents of the accumulation register again with saturation on overflow.
    The upper 8 bits of the result are returned to register VRT if the condition is met.
    The accumulation register is not modified.
}


\fxvinst{Move to accumulator byte fractional}{fxvmtacbf}{30}{%
    \For {$0 \le i < 16$}
        \State enable $\gets$ \Call{conditional}{C, VCR$_{i/16}$}

        \If { enable }
            \State $(\text{ACC})_{i/16} \gets (\text{VRA})_{i/16} \cdot 2^{8}$
        \EndIf
    \EndFor
}{%
    If enabled, the contents of elements in register VRA are copied to the accumulator aligned to the left.
}


\fxvinst{Multiply accumulate and save to accumulator byte fractional}{fxvmatacbfs}{61}{%
    \For {$0 \le i < 16$}
        \State $u \gets (\text{VRA})_{i/16}$
        \State $v \gets (\text{VRB})_{i/16}$
        \State $W \gets \Call{mult\_sat\_fract}{u, v, \unit[8]{bit}}$
        \State $M \gets \Call{add\_sat}{\text{ACC}_{i/16}, W, \unit[16]{bit}}$
        
        \State enable $\gets$ \Call{conditional}{C, VCR$_{i/16}$}

        \If { enable }
            \State $(\text{ACC})_{i/16} \gets M$
        \EndIf
    \EndFor
}{%
    Perform a multiply-add operation of registers VRA, VRB, and the accumulator using saturation and fractional arithmetics.
    If enabled, the result is stored in the accumulator.
}


\fxvinst{Multiply byte fractional saturating}{fxvmulbfs}{93}{%
    \For {$0 \le i < 16$}
        \State $u \gets (\text{VRA})_{i/16}$
        \State $v \gets (\text{VRB})_{i/16}$
        \State $W \gets \Call{mult\_sat\_fract}{u, v, \unit[8]{bit}}$
        
        \State enable $\gets$ \Call{conditional}{C, VCR$_{i/16}$}

        \If { enable }
            \State $(\text{VRT})_{i/16} \gets \Call{shift\_right}{M, 8}$
        \EndIf
    \EndFor
}{%
    Perform saturating multiplication in fractional representation of the contents of registers VRA and VRB.
    Store the truncated result to register VRT if enabled.
}


\fxvinst{Multiply and save to accumulator byte fractional saturating}{fxvmultacbfs}{125}{%
    \For {$0 \le i < 16$}
        \State $u \gets (\text{VRA})_{i/16}$
        \State $v \gets (\text{VRB})_{i/16}$
        \State $W \gets \Call{mult\_sat\_fract}{u, v, \unit[8]{bit}}$
        
        \State enable $\gets$ \Call{conditional}{C, VCR$_{i/16}$}

        \If { enable }
            \State $(\text{ACC})_{i/16} \gets M$
        \EndIf
    \EndFor
}{%
    Perform saturating multiplication in fractional representation of the contents of registers VRA and VRB.
    Store the result to the accumulator.
}


\fxvinst{Subtract byte fractional saturating}{fxvsubbfs}{349}{%
    \For {$0 \le i < 16$}
        \State $U \gets \Call{sign\_extend}{(\text{VRA})_{i/16}}$
        \State $V \gets \Call{sign\_extend}{- (\text{VRB})_{i/16}}$
        \State $W \gets \Call{add\_sat\_fract}{U, V, \unit[16]{bit}}$
        
        \State enable $\gets$ \Call{conditional}{C, VCR$_{i/16}$}

        \If { enable }
            \State $(\text{VRT})_{i/16} \gets \Call{shift\_right}{W, 8}$
        \EndIf
    \EndFor
}{%
    Add the contents of register VRA to the negated contents of register VRB using saturation.
    Store the truncated result to register VRT if enabled.
}


\fxvinst{Add accumulator and save to accumulator byte fractional}{fxvaddactacbf}{381}{%
    \For {$0 \le i < 16$}
        \State $U \gets \Call{sign\_extend}{(\text{VRA})_{i/16}}$
        \State $V \gets (\text{ACC})_{i/16}$
        \State $W \gets \Call{add\_sat\_fract}{U, V, \unit[16]{bit}}$
        
        \State enable $\gets$ \Call{conditional}{C, VCR$_{i/16}$}

        \If { enable }
            \State $(\text{ACC})_{i/16} \gets W$
        \EndIf
    \EndFor
}{%
    Add the sign extended contents of register VRA to the accumulator using saturating arithmetics and store the result to the accumulator.
}


\fxvinst{Add accumulator byte fractional saturating}{fxvaddacbfs}{413}{%
    \For {$0 \le i < 16$}
        \State $U \gets \Call{sign\_extend}{(\text{VRA})_{i/16}}$
        \State $V \gets (\text{ACC})_{i/16}$
        \State $W \gets \Call{add\_sat\_fract}{U, V, \unit[16]{bit}}$
        
        \State enable $\gets$ \Call{conditional}{C, VCR$_{i/16}$}

        \If { enable }
            \State $(\text{VRT})_{i/16} \gets \Call{shift\_right}{W, 8}$
        \EndIf
    \EndFor
}{%
    Add the sign extended contents of register VRA to the accumulator using saturating arithmetics.
    Store the truncated result to register VRT.
}


\fxvinst{Add byte fractional saturating}{fxvaddbfs}{477}{%
    \For {$0 \le i < 16$}
        \State $U \gets \Call{sign\_extend}{(\text{VRA})_{i/16}}$
        \State $V \gets \Call{sign\_extend}{(\text{VRB})_{i/16}}$
        \State $W \gets \Call{add\_sat\_fract}{U, V, \unit[16]{bit}}$
        
        \State enable $\gets$ \Call{conditional}{C, VCR$_{i/16}$}

        \If { enable }
            \State $(\text{VRT})_{i/16} \gets \Call{shift\_right}{W, 8}$
        \EndIf
    \EndFor
}{%
    Add the contents of registers VRA and VRB using saturating arithmetics.
    Store the result to register VRT.
}


\end{multicols}


\begin{multicols}{2}[%
    \subsection{Permute instructions}%
]

\fxvinst{Select}{fxvsel}{319}{%
    % algorithmic description
    \For {$0 \le i < 16$}
        \State $u \gets (\text{VRA})_{i/16}$
        \State $v \gets (\text{VRB})_{i/16}$
        
        \State enable $\gets$ \Call{conditional}{C, VCR$_{i/16}$}

        \If { enable }
            \State $(\text{VRT})_{i/16} \gets v$
        \Else
            \State $(\text{VRT})_{i/16} \gets u$
        \EndIf
    \EndFor
}{%
    % plain text description
    Use the contents of the condition register (VCR) to merge vectors VRA and VRB into the result vector VRT.
    The condition code selects which field of VCR to use.
    Select always merges bytewise.
    Hint: use in conjunction with the appropriate compare operation.
}


\fxvinst{Shift left halfword}{fxvshh}{316}{%
    % algorithmic description
    \For {$0 \le i < 8$}
        \State $u \gets (\text{VRA})_{i/8}$
        \State $(\text{VRT})_{i/8} \gets \Call{shift\_left}{u, VRB}$
    \EndFor
}{%
    % plain text description
    Shift the contents of register VRA left halfword-wise by the amount VRB.
    Here VRB is used as immediate value.
    Store the result in register VRT.
}


\fxvinst{Shift left byte}{fxvshb}{317}{%
    % algorithmic description
    \For {$0 \le i < 16$}
        \State $u \gets (\text{VRA})_{i/16}$
        \State $(\text{VRT})_{i/16} \gets \Call{shift\_left}{u, VRB}$
    \EndFor
}{%
    % plain text description
    Shift the contents of register VRA left byte-wise by the amount VRB.
    Here VRB is used as immediate value.
    Store the result in register VRT.
}
    

\fxvinst{Pack byte upper}{fxvpckbu}{239}{%
    \State $m \gets \Call{shift\_right}{\texttt{0xffff}, C}$
    \For {$0 \le i < 16$}
        \If { $i < 4$ }
            \State $u \gets \Call{shift\_right}{\text{VRA}_{2i/16}, C}$
            \State $\text{VRT}_{i/16} \gets \Call{bitwise\_and}{u, m}$
        \Else
            \State $u \gets \Call{shift\_right}{\text{VRB}_{2(i-4)/16}, C}$
            \State $\text{VRT}_{i/16} \gets \Call{bitwise\_and}{u, m}$
        \EndIf
    \EndFor
}{%
    Pack the upper half of fractional numbers of \unit[8-C]{bit} stored in halfword elements in registers VRA and VRB into register VRT, aligned to the right.
}


\fxvinst{Pack byte lower}{fxvpckbl}{255}{%
    \State $m \gets \Call{shift\_right}{\texttt{0xffff}, C}$
    \For {$0 \le i < 16$}
        \If { $i < 4$ }
            \State $u \gets \Call{shift\_left}{\text{VRA}_{2i/16}, 8-2C}$
            \State $v \gets \Call{shift\_right}{\text{VRA}_{2i+1/16}, 2C}$
            \State $w \gets \Call{bitwise\_or}{u, v}$
            \State $\text{VRT}_{i/16} \gets \Call{bitwise\_and}{w, m}$
        \Else
            \State $u \gets \Call{shift\_right}{\text{VRB}_{2(i-4)/16}, C}$
            \State $\text{VRT}_{i/16} \gets \Call{bitwise\_and}{u, m}$
        \EndIf
    \EndFor
}{%
    Pack the lower half of fractional numbers of \unit[8-C]{bit} stored in halfword elements in registers VRA and VRB into register VRT, aligned to the right.
}


\fxvinst{Unpack byte left}{fxvupckbl}{287}{%
    \State
}{%
    Reverse the packing operation performed by fxvpckbu and fxvpckbl and store the left half of the elements in register VRT.
}


\fxvinst{Unpack byte right}{fxvupckbr}{271}{%
    \State
}{%
    Reverse the packing operation performed by fxvpckbu and fxvpckbl and store the right half of the elements in register VRT.
}

\end{multicols}


\begin{multicols}{2}[%
    \subsection{Load/store instructions}
]

\fxvinstGPR{Load array indexed}{fxvlax}{492}{%
    \If { RA = 0 }
        \State $a \gets (\text{RB})$
    \Else
        \State $a \gets (\text{RA}) + (\text{RB})$
    \EndIf

    \For {$0 \le i < \text{NUM\_SLICES}$}
        \For {$0 \le j < 4$}
            \State $(\text{VRT})^{i}_{j\times 32 : (j+1)\times 32} \gets \Call{mem}{a, a+3}$
            \State $a \gets a + 4$
        \EndFor
    \EndFor
}{%
    The effective address is computed from general purpose registers RA and RB.
    If RA is zero, then the effective address is taken from RB.
    The contents of the vector register indicated by VRT is loaded starting from the memory location indicated by the effective address across the complete array of slices.
}


\fxvinstGPR{Store array indexed}{fxvstax}{508}{%
    \If { RA = 0 }
        \State $a \gets (\text{RB})$
    \Else
        \State $a \gets (\text{RA}) + (\text{RB})$
    \EndIf

    \For {$0 \le i < \text{NUM\_SLICES}$}
        \For {$0 \le j < 4$}
            \State $\Call{mem}{a, a+3} \gets (\text{VRT})^{i}_{j\times 32 : (j+1)\times 32}$
            \State $a \gets a + 4$
        \EndFor
    \EndFor
}{%
    The effective address is computed from general purpose registers RA and RB.
    If RA is zero, then the effective address is taken from RB.
    The contents of the vector register indicated by VRT is stored to memory starting from the location indicated by the effective address across the complete array of slices.
}


\fxvinstGPRc{Input indexed}{fxvinx}{236}{%
    \If { RA = 0 }
        \State $a \gets (\text{RB})$
    \Else
        \State $a \gets (\text{RA}) + (\text{RB})$
    \EndIf

    \State $\underline{u} \gets \Call{io}{a}$

    \For {$0 \le i < \text{NUM\_SLICES}$}
        \For {$0 \le j < 16$}
            \State enable $\gets$ \Call{conditional}{C, VCR$^i_{j/16}$}
            \If { enable }
                \State $\text{VRT}_{j/16}^i \gets \underline{u}_{j/16}^i$
            \EndIf
        \EndFor
    \EndFor
}{%
    The effective address is computed from general purpose registers RA and RB.
    If RA is zero, then the effective address is taken from RB.
    Data is loaded in parallel from the \gls{io} location referenced by the effective address and moved conditionally (if $C > 0$ it uses contents of condition register (VCR) for masking) to the destination register VRT.
}


\fxvinstGPRc{Output indexed}{fxvoutx}{252}{%
    \If { RA = 0 }
        \State $a \gets (\text{RB})$
    \Else
        \State $a \gets (\text{RA}) + (\text{RB})$
    \EndIf

    \State $\underline{u} \gets 0$

    \For {$0 \le i < \text{NUM\_SLICES}$}
        \For {$0 \le j < 16$}
            \State enable $\gets$ \Call{conditional}{C, VCR$^i_{j/16}$}
            \If { enable }
                \State $\underline{u}_{j/16}^i \gets \text{VRT}_{j/16}^i$
            \EndIf
        \EndFor
    \EndFor

    \State $\Call{io}{a} \gets \underline{u}$
}{%
    The effective address is computed from general purpose registers RA and RB.
    If RA is zero, then the effective address is taken from RB.
    The contents of registers VRT in all slices is written to the \gls{io} location referenced by the effective address.
    If $C > 0$ the write is masked by the condition register (VCR).
}

\end{multicols}

